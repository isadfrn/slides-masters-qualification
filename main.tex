\documentclass{beamer}
\usetheme[faculty=phil, logo=fibeamer/logo/logo]{fibeamer}
\usepackage[utf8]{inputenc}
\usepackage[main=brazilian]{babel}
\usepackage{ragged2e}
\usepackage{booktabs}
\usepackage{tabularx}
\usepackage{tikz}
\usetikzlibrary{calc, shapes, backgrounds}
\usepackage{amsmath, amssymb}
\usepackage{url}
\usepackage{listings}
\frenchspacing

\title{Um \textit{Framework} de Diretrizes para ampliar o ingresso e a permanência de talentos femininos em Programas de Capacitação em Cibersegurança}

\course{Programa de Pós-Graduação em Computação Aplicada}
\author{Aluna: Isabella de Freitas Nunes}
\advisor{Orientadora: Dra. Michelle Silva Wangham}

\begin{document}
  \frame[c]{\maketitle}

  \begin{frame}{Sumário}
    {%
      \setlength{\parskip}{0pt}%
      \setlength{\lineskip}{0pt}%
      \tableofcontents%
    }
  \end{frame}

  \section{Introdução}
  \begin{darkframes}
    \begin{frame}{}
      \justifying
      \begin{itemize}
        \item A escassez global de profissionais não é apenas numérica
              (\emph{headcount}), mas de habilidades (\emph{skills gap}).
        \item No Brasil, o déficit é de \textbf{230 mil profissionais}.
              O Decreto nº 12.573/2025 instituiu a E-Ciber, que exige a
              formação técnico-profissional em cibersegurança em larga escala.
      \end{itemize}
    \end{frame}

  \end{darkframes}

  \section{Leaky Pipeline e Gap de Gênero}

  \begin{frame}{O Fenômeno do Leaky Pipeline e o Gap de Gênero}
    \justifying
    \begin{itemize}
      \item Mulheres representam apenas \textbf{\~{}25\%} da força de trabalho,
            mesmo sendo mais escolarizadas.
      \item O \emph{leaky pipeline} mostra que a evasão feminina é
            desproporcional à medida que a competitividade e o nível técnico
            aumentam.
    \end{itemize}
  \end{frame}

  \section{Estudo de Caso: Hackers do Bem}

  \begin{darkframes}

    \begin{frame}{O Estudo de Caso: Programa Hackers do Bem}
      \justifying
      \begin{block}{O que é}
        Programa nacional de formação massiva, apoiado pelo
        MCTI/Softex/RNP/SENAI.
      \end{block}
      \bigskip
      \begin{alertblock}{O Funil}
        São mais de \textbf{150 mil inscritos}. As mulheres representam
        \textbf{22,45\%} na entrada, mas essa participação cai para cerca de
        \textbf{13\%} nas fases avançadas (síncronas e especializadas).
      \end{alertblock}
    \end{frame}

  \end{darkframes}

\end{document}
