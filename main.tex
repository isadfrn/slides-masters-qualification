\documentclass{beamer}
\usetheme[faculty=phil, logo=fibeamer/logo/logo]{fibeamer}
\usepackage[utf8]{inputenc}
\usepackage[main=brazilian]{babel}
\usepackage{csquotes}
\usepackage{ragged2e}
\usepackage{booktabs}
\usepackage{colortbl}
\usepackage{tabularx}
\usepackage{tikz}
\usetikzlibrary{calc, shapes, backgrounds}
\usepackage{amsmath, amssymb}
\usepackage{url}
\usepackage{listings}
\providecommand{\citeonline}[1]{\cite{#1}}
\frenchspacing
% Frame "capa" para iniciar cada section (mesmo fundo do darkframes).
% - fundo: igual ao `darkframes`
% - texto: branco (via cores do `darkframes`)
% - conteúdo: apenas 1 título, centralizado no eixo X
\newcommand{\sectioncover}[1]{%
  \begin{darkframes}%
    \begin{frame}[plain,c]
      \centering
      \vfill
      {\usebeamercolor[fg]{normal text}\usebeamerfont{title}\Huge\bfseries #1\par}
      \vfill
    \end{frame}%
  \end{darkframes}%
}

% Ativa automaticamente a capa no começo de cada seção:
\AtBeginSection{%
  \sectioncover{\insertsectionhead}%
}

% -------------------------- Capa
\title{Um \textit{Framework} de Diretrizes para ampliar o ingresso e a permanência de talentos femininos em Programas de Capacitação em Cibersegurança}
\course{Programa de Pós-Graduação em Computação Aplicada}
\author{Aluna: Isabella de Freitas Nunes}
\advisor{Orientadora: Dra. Michelle Silva Wangham}
% -------------------------- Começo do documento
\begin{document}
	% -------------------------- Sumário
	\frame[c]{\maketitle}
	\begin{frame}{Sumário}
		\tableofcontents
	\end{frame}
	% -------------------------- Introdução
	\section{Introdução}
		% -------------------------- Cenário Global
		\begin{frame}{Cenário Global}
			\begin{itemize}
				\justifying
				\item A cibersegurança consolidou-se como um pilar estratégico para a soberania nacional e estabilidade econômica global;
				\item Força de trabalho precisa crescer para atender à crescente demanda por proteção de dados;
				\item Aumento de apenas 0,1\% em relação a 2023;
				\item Força de trabalho desacelerando, considerando o aumento de 8,7\% entre 2022 e 2023;
			\end{itemize}
		\end{frame}
		% -------------------------- Sub-representação feminina
		\begin{frame}{Sub-representação feminina}
			\begin{itemize}
				\justifying
				\item  A participação feminina na força de trabalho global está entre 20\% e 25\%;	
				\item 26\% dos profissionais com menos de 30 anos;
				\item 13\% da força de trabalho acima de 60 anos;
				\item Em média, mais escolarizadas: 52\% possuem pós-graduação, comparado a 44\% de seus pares masculinos;
				\item Exercem 17\% dos cargos de liderança executiva;
				\item Recebem cerca de 12\% a menos que homens nas posições de direções de liderança;
			\end{itemize}
		\end{frame}
		% -------------------------- Cenário brasileiro
		\begin{frame}{Cenário brasileiro}
			\begin{itemize}
				\justifying
				\item  Decreto n.º 12.573/2025 (BRASIL, 2025), que institui a nova Estratégia Nacional de Cibersegurança (E-Ciber);
				\item Programa Hackers do Bem;
				\item Qualificar mais de 30 mil profissionais por meio de um ecossistema de ensino gratuito, escalável e fundamentado em gamificação e atividades práticas simuladas;
			\end{itemize}
		\end{frame}
		% -------------------------- Problema de Pesquisa
		\begin{frame}{Problema de Pesquisa}
			\begin{itemize}
				\justifying
				\item A literatura aponta para um fenômeno persistente de “vazamento de talentos” (\textit{leaky pipeline}), onde as taxas de evasão femininas superam as masculinas à medida que o nível técnico e a competitividade aumentam;
				\item A retenção é influenciada por barreiras estruturais, culturais e pedagógicas;
			\end{itemize}
		\end{frame}
		% -------------------------- Questão de Pesquisa
		\begin{frame}{Questão de Pesquisa}
			\justifying
			Como programas de capacitação em cibersegurança com capacidade de atendimento em larga escala, como o Hackers do Bem, podem ser customizados para ampliar o ingresso e a permanência de talentos femininos?
		\end{frame}
		% -------------------------- Solução Proposta
		\begin{frame}{Solução Proposta}
			\begin{itemize}
				\justifying
				\item Desenvolvimento de um \textit{framework} de diretrizes voltado à inclusão e retenção de mulheres em programas de capacitação em cibersegurança;
				\item  Mineração de dados educacionais de participantes do Programa Hackers do Bem com a análise qualitativa da percepção das alunas;
			\end{itemize}
		\end{frame}
		% -------------------------- Delimitação de Escopo
		\begin{frame}{Delimitação de Escopo}
			\begin{itemize}
				\justifying
				\item  Análise dos dados das turmas do Programa Hackers do Bem realizadas entre 2024 e 2026;
				\item Não faz parte do escopo a alteração do código-fonte da plataforma de ensino ou a intervenção direta nas turmas em andamento, caracterizando-se como um estudo observacional e propositivo;
			\end{itemize}
		\end{frame}
		% -------------------------- Justificativa
		\begin{frame}{Justificativa}
			\begin{itemize}
				\justifying
				\item Diversidade em equipes de segurança resulta em maior eficiência operacional e melhor desempenho na resolução de problemas complexos;
				\item Investigar os fatores que influenciam a permanência de mulheres em programas de capacitação em cibersegurança;
			\end{itemize}
		\end{frame}
		% -------------------------- Objetivos
		\begin{frame}{Objetivo Geral}
			\begin{block}{}
				\justifying
				Definir e avaliar um conjunto de diretrizes para o aumento da atração e permanência feminina em programas de capacitação em cibersegurança.
			\end{block}
		\end{frame}
		% -------------------------- Objetivos Específicos
		\begin{frame}{Objetivos Específicos}
			\begin{block}{}
				\justifying
				Mapear as estratégias de retenção e barreiras de gênero documentadas na literatura, por meio de Revisão Sistemática da Literatura (RSL).
			\end{block}
			\begin{block}{}
				\justifying
				Identificar os pontos críticos de evasão feminina do Programa Hackers do Bem, por meio de uma análise quantitativa do funil de progressão do programa e de uma análise qualitativa conduzida por meio de entrevistas com as participantes do programa.
			\end{block}
		\end{frame}
		% -------------------------- Objetivos Específicos
		\begin{frame}{Objetivos Específicos}
			\begin{block}{}
				\justifying
				Avaliar a correlação entre os mecanismos de gamificação (Pontos de Experiência (XP), Ranking), outras estratégias de engajamento e o desempenho das participantes do programa, por meio da análise dos dados do programa Hackers do Bem.
			\end{block}
			\begin{block}{}
				\justifying
				Elaborar o Framework de Diretrizes para Inclusão e Retenção, fundamentado nos diagnósticos quantitativos e qualitativos.
			\end{block}
			\begin{block}{}
				\justifying
				Avaliar a viabilidade técnica e a pertinência pedagógica das diretrizes propostas junto a um painel de especialistas.
			\end{block}
		\end{frame}
		% -------------------------- Objetivos Específicos
		\begin{frame}{Objetivos Específicos}
			\begin{block}{}
				\justifying
				Avaliar a percepção de valor e o impacto potencial das diretrizes na intenção de permanência das alunas do Programa Hackers do Bem, por meio de questionários de avaliação com as participantes do programa.
			\end{block}
		\end{frame}
		% -------------------------- Metodologia da Pesquisa e Procedimentos metodológicos
		\begin{frame}{Metodologia da Pesquisa e Procedimentos metodológicos}
			\justifying
			\begin{itemize}
				\item RSL
				\item Análise de trabalhos relacionados
				\item Coleta de Dados
				\item Análise de Dados
				\item Concepção e Modelagem do Framework
				\item Avaliação da Proposta
			\end{itemize}
		\end{frame}
	% -------------------------- Fim da Introdução
	% -------------------------- Fundamentação Teórica
	\section{Fundamentação Teórica}
		% -------------------------- Barreiras de Entrada e o Fenômeno do Leaky Pipeline
		\begin{frame}{Barreiras de Entrada e o Fenômeno do \textit{Leaky Pipeline}}
			\justifying
			O conceito de \textit{Leaky Pipeline} (vazamento de talentos) é amplamente utilizado na literatura para descrever a perda progressiva e desproporcional da participação feminina ao longo das etapas de formação acadêmica e transição para a carreira.
		\end{frame}
		% -------------------------- Barreiras de Entrada e o Fenômeno do Leaky Pipeline
		\begin{frame}{Barreiras de Entrada e o Fenômeno do \textit{Leaky Pipeline}}
			\begin{itemize}
				\justifying
				\item Iniciam-se no acesso desigual à STEM;
				\item Escassez de modelos de referência (\textit{role models});
				\item Estereótipos culturais;
				\item Dissonância de identidade para jovens mulheres;
			\end{itemize}
		\end{frame}
		% -------------------------- Metodologias de Ensino e Gamificação
		\begin{frame}{Metodologias de Ensino e Gamificação}
			\begin{itemize}
				\justifying
				\item Gamificação e Aprendizagem Baseada em Jogos (GBL);
				\item \textit{Capture The Flag};
				\item Competições;
				\item Listas de classificação pública (\textit{leaderboards});
			\end{itemize}
		\end{frame}
		% -------------------------- O Programa Hackers do Bem
		\begin{frame}{O Programa \textit{Hackers} do Bem}
			\begin{itemize}
				\justifying
				\item Formação de recursos humanos qualificados em cibersegurança e privacidade
				\item Atender um público massivo
				\item Níveis progressivos
				\begin{itemize}
					\justifying
					\item Nivelamento
					\item Básico
					\item Fundamental
					\item Especializado
					\item Residência Tecnológica
				\end{itemize}
			\end{itemize}
		\end{frame}
		% -------------------------- Estrutura da Formação
		\begin{frame}{Estrutura da Formação}
			\begin{center}
				\includegraphics[width=1\textwidth]{./images/estrutura.png}
			\end{center}
		\end{frame}
		% -------------------------- Mecânica de Pontuação e Emblemas
		\begin{frame}{Mecânica de Pontuação e Emblemas}
			\begin{itemize}
				\justifying
				\item A unidade fundamental de progresso no AVA do Programa Hackers do Bem é o Ponto de Experiência (XP)
				\item  Cada interação do aluno, como o consumo de videoaulas, a leitura de e-books e a realização de quizzes de fixação
				\item O acúmulo destes pontos desbloqueia emblemas (badges) que atuam como indicadores visuais de senioridade e progressão na trilha formativa
			\end{itemize}
		\end{frame}
		% -------------------------- Mecânica de Pontuação e Emblemas
		\begin{frame}{Mecânica de Pontuação e Emblemas}
			\begin{center}
				\includegraphics[width=1\textwidth]{./images/emblemas.png}
			\end{center}
		\end{frame}
		% -------------------------- O Funil de Ranqueamento e Classificação
		\begin{frame}{O Funil de Ranqueamento e Classificação}
			\begin{center}
				\includegraphics[width=1\textwidth]{./images/pontuacoes.png}
			\end{center}
		\end{frame}
		% -------------------------- Ações de Engajamento e Eventos Práticos
		\begin{frame}{Ações de Engajamento e Eventos Práticos}
			\begin{itemize}
				\justifying
				\item \textit{CyberGames}
				\item CTFs (CTF Para Elas)
				\item Hackathons
				\item Workshops
				\item Hub \textit{Hackers} do Bem
			\end{itemize}
		\end{frame}
	% -------------------------- Fim da Fundamentação Teórica
	% -------------------------- Trabalhos Relacionados
	\section{Trabalhos Relacionados}
		% -------------------------- PICO
		\begin{frame}{PICO}
			\vspace{0.3cm}
			\centering
			\begin{footnotesize}
				\begin{tabular}{|l|p{6.5cm}|}
					\hline
					\textbf{Elemento} & \textbf{Definição na Pesquisa} \\
					\hline
					\textbf{População} & Meninas, mulheres, estudantes e profissionais em cibersegurança. \\
					\hline
					\textbf{Intervenção} & Projetos, programas, ações de inclusão, capacitação, mentoria, oficinas, eventos e \textit{hackathons}. \\
					\hline
					\textbf{Comparação} & Projetos em cibersegurança (gerais/com recorte de gênero). \\
					\hline
					\textbf{Resultado (\textit{Outcome})} & Aumento do número de ações voltadas às mulheres na área de cibersegurança. \\
					\hline
				\end{tabular}
			\end{footnotesize}
			\vspace{0.2cm}
			\raggedright
		\end{frame}
		% -------------------------- String de busca
		\begin{frame}{\textit{String} de busca}
			\begin{exampleblock}{}
				\justifying
				\textit{(\enquote{women} OR \enquote{girls} OR \enquote{female} OR \enquote{gender}) AND (\enquote{cybersecurity} OR \enquote{digital security} OR \enquote{information security} OR \enquote{cyber security}) AND (\enquote{inclusion} OR \enquote{outreach} OR \enquote{mentor} OR \enquote{train} OR \enquote{empower} OR \enquote{project} OR \enquote{hackathon} OR \enquote{workshop} OR \enquote{bootcamp})}
			\end{exampleblock}
		\end{frame}
		% -------------------------- Repositórios
		\begin{frame}{Repositórios}
			\begin{itemize}
				\justifying
				\item Livros e artigos completos publicados em periódicos ou eventos
				\item Período de 2020 a 2025
				\item Bases de dados
				\begin{itemize}
					\justifying
					\item IEEE Xplore
					\item Scopus
					\item Google Scholar
					\item ACM
				\end{itemize}
			\end{itemize}
		\end{frame}
		%-------------------------- Etapas da Seleção
		\begin{frame}{Etapas da Seleção}
			\begin{itemize}
				\justifying
				\item Primeira fase: triagem por título
					\begin{itemize}
						\justifying
						\item Inclusão:
							\begin{itemize}
								\justifying
								\item O artigo completo está disponível
								\item O artigo aborda treinamento em cibersegurança ou a inclusão de mulheres no setor
							\end{itemize}
						\item Exclusão:
							\begin{itemize}
								\justifying
								\item Se o título não é conclusivo para inclusão, o abstract foi lido, se o abstract não atende os critérios de inclusão dessa fase, o artigo é excluído
							\end{itemize}
					\end{itemize}
			\end{itemize}
		\end{frame}
		%-------------------------- Etapas da Seleção
		\begin{frame}{Etapas da Seleção}
			\begin{itemize}
				\justifying
				\item Segunda fase: triagem por abstract
					\begin{itemize}
						\justifying
						\item Inclusão:
							\begin{itemize}
								\justifying
								\item Conexão direta entre o ensino de cibersegurança e
								estratégias de inclusão de mulheres
								\item Análise de gênero
							\end{itemize}
						\item Exclusão:
							\begin{itemize}
								\justifying
								\item Estudos puramente técnicos
								\item Sem dados de recorte de gênero
							\end{itemize}
					\end{itemize}
			\end{itemize}
		\end{frame}
		%-------------------------- Etapas da Seleção
		\begin{frame}{Etapas da Seleção}
			\begin{itemize}
				\justifying
				\item Terceira fase: leitura do texto completo
					\begin{itemize}
						\justifying
						\item Inclusão:
							\begin{itemize}
								\justifying
								\item O artigo relata alguma estratégia utilizada para ampliar o ingresso e a permanência de talentos femininos em programas de treinamento em cibersegurança?
							\end{itemize}
						\item Exclusão:
							\begin{itemize}
								\justifying
								\item O artigo NÃO relata alguma estratégia utilizada para ampliar o ingresso e a permanência de talentos femininos em programas de treinamento em cibersegurança?
							\end{itemize}
					\end{itemize}
			\end{itemize}
		\end{frame}
		% -------------------------- Artigos selecionados
		\begin{frame}{Artigos selecionados}
			\vspace{0.3cm}
			\centering
			\begin{footnotesize}
				\begin{tabular}{l|c|c|c|c}
					\toprule
					\textbf{Base de dados} & \textbf{Identificação} & \textbf{Primeira fase} & \textbf{Segunda fase} & \textbf{Seleção final} \\
					\midrule
					Google Scholar & 50  & 29 & 6  & 1 \\
					Scopus         & 54  & 23 & 5  & 1 \\
					IEEE Xplore    & 342 & 32 & 11 & 1 \\
					ACM            & 215 & 30 & 10 & 6 \\
					\midrule
					\textbf{Total} & \textbf{661} & \textbf{114} & \textbf{32} & \textbf{9} \\
					\bottomrule
				\end{tabular}
			\end{footnotesize}
			\vspace{0.2cm}
			\raggedright
		\end{frame}
		% -------------------------- Distribuição temporal dos trabalhos selecionados
		\begin{frame}{Distribuição temporal dos trabalhos selecionados}
			\begin{center}
				\includegraphics[width=1\textwidth]{./images/distribuicao-temporal.png}
			\end{center}
		\end{frame}
		% -------------------------- Análise Multidimensional dos Trabalhos
		\begin{frame}{Análise Multidimensional dos Trabalhos}
			\begin{itemize}
				\item Foco em Gênero e Barreiras (QP1)
				\item Abordagens Pedagógicas (QP2)
				\item Colaboração e Autoeficácia (QP3)
				\item Escalabilidade e Alcance
				\item Avaliação
			\end{itemize}
		\end{frame}
		% --------------------------Análise Multidimensional dos Trabalhos
		\begin{frame}{Análise Multidimensional dos Trabalhos}
			\vspace{0.2cm}
			\centering
			\vspace{0.25cm}
			\resizebox{\textwidth}{!}{%
			\begin{footnotesize}
				\begin{tabular}{l|c|c|c|c|c}
					\toprule
					\textbf{Trabalho} & \textbf{\shortstack{Gênero\\(QP1)}} & \textbf{\shortstack{Pedagogia\\(QP2)}} & \textbf{\shortstack{Colab.\\(QP3)}} & \textbf{\shortstack{Escala-\\bilidade}} & \textbf{\shortstack{Avali-\\ação}} \\
					\midrule
					Costa et al. (2025) & 5 & 5 & 3 & 4 & 5 \\
					Musuva et al. (2025) & 4 & 3 & 5 & 3 & 4 \\
					Tshekiso et al. (2025) & 3 & 2 & 3 & 5 & 4 \\
					Benson, Chiacchio e Fraczek (2025) & 5 & 2 & 5 & 3 & 3 \\
					Costa et al. (2023) & 5 & 5 & 4 & 3 & 4 \\
					Casey et al. (2023) & 4 & 5 & 4 & 2 & 4 \\
					Thomas et al. (2024) & 4 & 5 & 2 & 2 & 3 \\
					Rahman, Billionniere e Subhedar (2022) & 5 & 3 & 4 & 3 & 4 \\
					Hogan et al. (2025) & 2 & 3 & 5 & 5 & 4 \\
					\bottomrule
				\end{tabular}
			\end{footnotesize}%
			}
			\vspace{0.2cm}
			\raggedright
		\end{frame}
		
		% -------------------------- Análise Multidimensional dos Trabalhos
		\begin{frame}{Análise Multidimensional dos Trabalhos}
			\begin{center}
				\includegraphics[width=\textwidth, height=0.7\textheight, keepaspectratio]{./images/analise-multidimensional.png}
			\end{center}
		\end{frame}
		% -------------------------- Comparação e Lacunas Identificadas
		\begin{frame}{Comparação e Lacunas Identificadas}
			\begin{itemize}
				\justifying
				\item Embora existam iniciativas robustas em isolamento, há uma carência de modelos que integrem simultaneamente pedagogias de inclusão profunda (como narrativas e \textit{scaffolding}) com estratégias de larga escala (como plataformas \textit{EAD} massivas).
			\end{itemize}
		\end{frame}
		% -------------------------- Comparação e Lacunas Identificadas
		\begin{frame}{Comparação e Lacunas Identificadas}
			\centering
			\resizebox{\textwidth}{!}{%
			\begin{footnotesize}
				\begin{tabular}{l|c|c|c|c|>{\raggedright\arraybackslash}p{3.4cm}}
					\toprule
					\rowcolor[HTML]{EFEFEF}
					\textbf{Trabalho} & \textbf{Foco Gênero} & \textbf{Escalab.} & \textbf{Pedag. Incl.} & \textbf{Colab./Ment.} & \textbf{Limitação principal (gap)} \\
					\midrule
					Costa et al. (2025) & Alta & Média & Alta & Média & Intervenção pontual, falta dados de carreira a longo prazo. \\
					Musuva et al. (2025) & Média & Média & Média & Alta & Modelo presencial intensivo, difícil de escalar massivamente. \\
					Tshekiso et al. (2025) & Baixa & Alta & Baixa & Média & Foco em infraestrutura/CTF padrão, sem pedagogia de gênero. \\
					Benson, Chiacchio e Fraczek (2025) & Alta & Baixa & Baixa & Alta & Estudo teórico/qualitativo, sem plataforma de ensino aplicada. \\
					Costa et al. (2023) & Alta & Média & Alta & Alta & Foco em atração (Ensino Médio), não em formação profissional. \\
					Casey et al. (2023) & Alta & Baixa & Alta & Alta & Depende de mediação humana intensa (workshops presenciais). \\
					Thomas et al. (2024) & Alta & Baixa & Alta & Baixa & Estudo de caso pequeno (N=24), foco em suporte individual. \\
					Rahman, Billionniere e Subhedar (2022) & Alta & Baixa & Média & Alta & Foco específico em retorno ao trabalho, escala de workshop. \\
					Hogan et al. (2025) & Baixa & Alta & Média & Alta & Analisa times mistos \enquote{de sucesso}, viés de sobrevivência. \\
					\midrule
					\rowcolor[HTML]{C0C0C0}
					\textbf{Esta Dissertação} & \textbf{Alta} & \textbf{Alta} & \textbf{Alta} & \textbf{Média} & \textbf{Avaliação em ambiente real massivo (\textit{Hackers do Bem}).} \\
					\bottomrule
				\end{tabular}%
			\end{footnotesize}%
			}
			\vspace{0.2cm}
			\raggedright
			{\tiny Fonte: Elaborado pela autora}
		\end{frame}
	% -------------------------- Fim dos Trabalhos Relacionados
	% -------------------------- Proposta
	\section{Proposta}
		% -------------------------- Framework de Diretrizes para Retenção
		\begin{frame}{Framework de Diretrizes para Retenção}
			\begin{itemize}
				\justifying
				\item \textbf{Eixo 1: Contextualização Narrativa:} Propõe diretrizes para a camada de apresentação do conteúdo, orientando a transição de exercícios técnicos abstratos para narrativas investigativas com propósito social, visando reduzir a barreira cultural de entrada.
			\end{itemize}
		\end{frame}
		% -------------------------- Framework de Diretrizes para Retenção
		\begin{frame}{Framework de Diretrizes para Retenção}
			\begin{itemize}
				\justifying
				\item \textbf{Eixo 2: Mecânicas de Avaliação Inclusiva:} Visa reformular a interpretação das métricas de gamificação (\textit{XP} e Ranking). A proposta estabelece diretrizes para valorizar a \enquote{Eficiência de Engajamento} (progresso individual) em detrimento da pura comparação competitiva pública.
			\end{itemize}
		\end{frame}
		% -------------------------- Framework de Diretrizes para Retenção
		\begin{frame}{Framework de Diretrizes para Retenção}
			\begin{itemize}
				\justifying
				\item \textbf{Eixo 3: Suporte Social Escalável:} Estabelece regras para a criação de redes de suporte em ambientes virtuais, operacionalizando a mentoria e o suporte de pares de forma escalável dentro da plataforma.
			\end{itemize}
		\end{frame}
		% -------------------------- Frente 1: Mineração de Dados (O “Onde” e o “Quando”)
		\begin{frame}{Frente 1: Mineração de Dados (O “Onde” e o “Quando”)}
			\begin{itemize}
				\justifying
				\item Análise Volumétrica
				\item Correlação de Desempenho
			\end{itemize}
		\end{frame}
		% -------------------------- Frente 2: Levantamento de Percepção (O “Porquê”)
		\begin{frame}{Frente 2: Levantamento de Percepção (O “Porquê”)}
			\begin{itemize}
				\justifying
				\item Autoeficácia em Cibersegurança
				\item Percepção de Competitividade
				\item Pertencimento
			\end{itemize}
		\end{frame}
		% -------------------------- Frente 2: Levantamento de Percepção (O “Porquê”)
		\begin{frame}{Frente 2: Levantamento de Percepção (O “Porquê”)}
			\begin{itemize}
				\justifying
				\item Autoeficácia em Cibersegurança
				\item Percepção de Competitividade
				\item Pertencimento
			\end{itemize}
		\end{frame}
	% -------------------------- Fim da Proposta
\end{document}